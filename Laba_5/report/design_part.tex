\chapter{Конструкторская часть}
\section{Используемые структуры данных и классы}
\begin{enumerate}
	\item \textbf{Класс \texttt{Matrix}} обеспечивает хранение и управление матрицами различных типов данных. Основные методы класса:
	\begin{itemize}
		\item \texttt{print} --- вывод матрицы на экран;
		\item \texttt{zero} --- заполнение матрицы нулевыми значениями;
	\end{itemize}
	\item \textbf{Основные переменные}:
	\begin{itemize}
		\item \texttt{D} --- матрица расстояний между узлами графа;
		\item \texttt{T} --- матрица феромонов;
		\item \texttt{Q} --- квота феромона;
		\item \texttt{N} --- количество узлов (городов) графа и количество муравьёв.
	\end{itemize}
\end{enumerate}

\section{Инициализация}
\begin{itemize}
	\item Чтение входных данных из файла. Файл содержит описание графа, включая количество узлов и матрицу расстояний \texttt{D}.
	\item Заполнение начальной матрицы феромонов \texttt{T}. Феромоны инициализируются минимальными значениями для всех рёбер.
	\item Установка параметров алгоритма, таких как \texttt{a} (влияние расстояния), \texttt{b} (влияние феромона), и \texttt{q} (коэффициент испарения феромонов).
\end{itemize}

\section{Основной цикл работы алгоритма}
Алгоритм выполняется в течение заданного количества итераций (дней). Каждая итерация включает шаги, приведённые ниже:

\subsection*{Построение маршрутов муравьями}
\begin{itemize}
	\item Каждый муравей начинает маршрут из случайного узла графа(каждый муравей из разного узла).
	\item На каждом шаге муравей выбирает следующий узел на основе вероятностей, которые вычисляются с учётом:
	\begin{itemize}
		\item текущего уровня феромонов на рёбрах;
		\item обратной величины расстояния до следующего узла.
	\end{itemize}
\end{itemize}

\subsection*{Обновление локальной матрицы феромонов}
\begin{itemize}
	\item После завершения маршрута каждого муравья обновляется локальная матрица феромонов на основе длины маршрута.
	\item Количество феромона, добавляемое на ребро, обратно пропорционально длине маршрута.
\end{itemize}

\subsection*{Обновление глобальной матрицы феромонов}
\begin{itemize}
	\item По завершении итерации выполняется глобальное обновление феромонов.
	\item Феромоны на каждом ребре испаряются с коэффициентом $q$.
	\item Значение $\tau_{ij}$ ограничивается минимальным уровнем, чтобы предотвратить полное исчезновение феромонов.
\end{itemize}

\section*{Выбор лучших маршрутов}
\begin{itemize}
	\item По окончании каждой итерации сохраняется маршрут с минимальной длиной.
	\item Если используется режим с элитными муравьями, феромоны дополнительно обновляются с учётом лучших маршрутов за всю историю работы алгоритма.
\end{itemize}

\section{Функции}

\section*{Функция \texttt{calculate\_median}}
Вычисляет медиану заданного вектора целых чисел.
\begin{itemize}
	\item Входные данные: вектор \texttt{v}.
	\item Алгоритм:
	\begin{enumerate}
		\item сортирует элементы вектора;
		\item если количество элементов нечётное, возвращает центральный элемент;
		\item если чётное, возвращает среднее арифметическое двух центральных элементов.
	\end{enumerate}
	\item Возвращаемое значение: медиана вектора.
\end{itemize}

\section*{Функция \texttt{calculate\_mean}}
Вычисляет среднее арифметическое элементов вектора.
\begin{itemize}
	\item Входные данные: вектор \texttt{v}.
	\item Алгоритм:
	\begin{enumerate}
		\item суммирует все элементы вектора;
		\item делит сумму на количество элементов.
	\end{enumerate}
	\item Возвращаемое значение: среднее арифметическое.
\end{itemize}

\section*{Функция \texttt{show\_route}}
Выводит маршрут, пройденный муравьём, и длины рёбер.
\begin{itemize}
	\item Входные данные: указатель на вектор \texttt{nodes}, содержащий последовательность узлов маршрута.
	\item Алгоритм:
	\begin{enumerate}
		\item для каждой пары последовательных узлов выводит их и длину ребра между ними;
		\item если включён режим цикла, также выводит ребро от последнего узла к первому.
	\end{enumerate}
	\item Вывод: маршрут и его длины.
\end{itemize}

\section*{Функция \texttt{get\_length}}
Вычисляет длину маршрута.
\begin{itemize}
	\item Входные данные: указатель на вектор \texttt{nodes}.
	\item Алгоритм:
	\begin{enumerate}
		\item суммирует длины всех рёбер маршрута;
		\item если включён режим цикла, добавляет длину ребра от последнего узла к первому.
	\end{enumerate}
	\item Возвращаемое значение: длина маршрута.
\end{itemize}

\section*{Функция \texttt{count\_Q}}
Вычисляет общее количество феромона \texttt{Q}, используемого для обновлений.
\begin{itemize}
	\item Алгоритм:
	\begin{enumerate}
		\item суммирует все расстояния в матрице \texttt{D};
		\item делит сумму на $N$ (или $N-1$, в зависимости от режима).
	\end{enumerate}
	\item Возвращаемое значение: значение \texttt{Q}.
\end{itemize}

\section*{Функция \texttt{upgrade\_pheromone}}
Обновляет матрицу феромонов на основе лучших маршрутов(используется только в модернизированном алгоритме).
\begin{itemize}
	\item Входные данные: указатель на вектор лучших маршрутов \texttt{routes}, длина лучшего маршрута \texttt{L\_best}.
	\item Алгоритм:
	\begin{itemize}
		\item для каждого маршрута из \texttt{routes} добавляет к соответствующим рёбрам феромон, пропорциональный $Q / L\_best$.
	\end{itemize}
\end{itemize}

\clearpage
