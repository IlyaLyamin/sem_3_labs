\chapter{Исследовательская часть}

\textbf{Цель исследования} --- выявление наиболее подходящих параметров для решения задачи коммивояжёра в исследуемых алгоритмах. По результатам параметризации представленных в приложении \ref{add} можно выделить несколько случаев наиболее выгодных параметров в соответствии со следующими задачами: наиболее быстрое выявление кратчайшего маршрута и выявление наиболее точного наименьшего маршрута. Реализации с параметрами ($a = 0.45,  q = 0.95$), ($a = 0.7, q = 0.45$), ($a = 0.7, q = 0.7$), ($a = 0.7, q = 0.95$), ($a = 0.95, q = 0.45$), находит наилучший маршрут за наименьшее количество циклов, то есть при пяти попытках запуска программы в среднем хотя бы один путь будет наименьшим. Но даже при 800 итерациях, при параметрах ($a = 0.45,  q = 0.95$), ($a = 0.7, q = 0.45$), ($a = 0.7, q = 0.7$), ($a = 0.95, q = 0.45$), будет находиться как минимум один не оптимальный маршрут. Реализация с параметрами $a = 0.7, b = 0.3, q = 0.95$ показала наиболее высокую точность. При лучшем маршруте длинной в 275, среднее по пяти попыткам этой реализации составило 277 при 200 итерациях, 276 при 500 итерациях и 275 при 800 итерациях. Следовательно данные параметры наиболее подходящие для выявления наилучшего маршрута с наибольшей точностью, но реализации с такими параметрами необходимо большее количество циклов. 

\section{Вывод}

Провели исследование параметризации реализованного алгоритма и нашли два класса параметров наиболее точно соответствующих поставленным задачам.

\clearpage
