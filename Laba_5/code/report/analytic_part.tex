\chapter{Аналитическая часть}

\section{Алгоритм муравья}
\textbf{Муравьиный алгоритм} --- это метаэвристический алгоритм для решения задач комбинаторной оптимизации, таких как задача коммивояжёра. Основной принцип работы алгоритма основан на моделировании поведения муравьёв, которые оставляют феромоны на пути между узлами, обозначая тем самым предпочтительные маршруты.

\textbf{Муравьиный алгоритм с элитными муравьями} --- данный алгоритм является модификацией предыдущего. Его отличие заключается в добавлении дополнительного феромона на рёбра/грани, входящие в наилучшие маршруты. Этот феромон добавляется в фазе $"$ночи$"$, когда обновляется весь феромон.

\section{Основные понятия}

\begin{itemize}
	\item \textbf{Феромоны}: Муравьи оставляют на рёбрах графа вещество --- феромон (обозначается как $\tau$), уровень которого влияет на вероятность выбора данного пути.
	\item \textbf{Эвристическая информация}: Дополнительные данные о предпочтительности перехода, такие как обратное расстояние между двумя городами: $\eta_{ij} = \frac{1}{d_{ij}}$, где $d_{ij}$ --- расстояние между городами $i$ и $j$.
	\item \textbf{Испарение феромонов}: Со временем уровень феромонов на рёбрах уменьшается, что позволяет алгоритму избегать зацикливания на субоптимальных решениях.
	\item \textbf{Комбинированная вероятность}: Выбор следующего города для посещения основан на сочетании уровня феромонов и эвристической информации.
\end{itemize}

\section{Алгоритм работы}

Алгоритм работает по описанию ниже.

\begin{enumerate}
	\item \textbf{Инициализация}:
	\begin{itemize}
		\item задаётся начальный уровень феромонов $\tau_{ij}$ для всех рёбер графа;
		\item определяются параметры алгоритма --- \(\alpha\), \(\beta\), \(\rho\) и количество муравьёв.
	\end{itemize}
	
	\item \textbf{Построение маршрута}.	Каждый муравей строит маршрут, постепенно посещая все города. Выбор следующего города $j$ осуществляется на основе вероятности:
	\begin{equation}
		P_{ij} = 
		\begin{cases}\label{for:Lev}
			\frac{\tau_{ij}^\alpha \cdot \eta_{ij}^\beta}{\sum_{k \in \text{непосещённые}} \tau_{ik}^\alpha \cdot \eta_{ik}^\beta}, & j \in \text{непосещённые} \\
			0 &
		\end{cases}
	\end{equation}
	где  $\tau_{ij}$ --- уровень феромонов на ребре $ij$,  $\eta_{ij}$ --- эвристическая информация,  $\alpha$ и $\beta$ --- параметры, управляющие влиянием феромонов и эвристической информации соответственно.
	
	\item \textbf{Обновление феромонов}:
	После того как все муравьи завершили построение маршрутов, уровни феромонов обновляются:
	\begin{equation}
		\tau_{ij} = (1 - \rho) \cdot \tau_{ij} + \Delta \tau_{ij},
	\end{equation}
	где $\rho$ --- коэффициент испарения феромонов, $\Delta \tau_{ij}$ --- добавка феромонов на основании качества маршрутов, использующих ребро $ij$.
	\newline Важно, что уровень феромонов на ребре никогда не опускается ниже предопрелённой константы $\varepsilon > 0$, тем самым обеспечивая вероятность выбора ребра, даже если оно находится на неразработанном маршруте.
	\par
	Добавка феромонов вычисляется следующим образом:
	\begin{equation}
		\Delta \tau_{ij} = \sum_{k} \frac{Q}{L_k},
	\end{equation}
	где $Q$ --- параметр алгоритма, а $L_k$ --- длина маршрута $k$-го муравья.
	
	В модернизации алгоритма выделяется дополнительный феромон на рёбрах, которые входят в лучшие маршруты.\par
		
	\item \textbf{Повторение}:
	Шаги построения маршрутов и обновления феромонов повторяются, пока не будет достигнут критерий остановки (например, заданное число итераций(дней)).
\end{enumerate}


\clearpage
