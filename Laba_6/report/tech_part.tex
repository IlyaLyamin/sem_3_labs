\chapter{Технологическая часть}

\section{Выбор средств реализации}
Для программной реализации алгоритма использовалась среда разработки Visual Studio
2022, язык программирования, на котором была выполнена реализации алгоритмов — C++.
Для компиляции кода использовался компилятор MSVC. Исследование проводилось на ноут
буке (64–разрядная операционная система, процессор x64, частота процессора 3.1 ГГц, модель
процессора 12th Gen Intel(R) Core(TM) i5-12500H, оперативная память 16 ГБ) \par
\section{Реализация дерева}
В листинге~\ref{list1} представлена программная реализация \textbf{АВЛ-дерева}.
\begin{lstlisting}[label = list1, caption = Программная реализация АВЛ-дерева с интерфейсом для пользователя]
#include <cstring> 
#include <iostream>
#include <queue>
#include <iomanip>
using namespace std;


struct Node {
	Node* L = nullptr;
	Node* R = nullptr;
	char* content = nullptr;
	int height;
	
	Node(char* el) {
		Node* L = new Node;
		Node* R = new Node;
		content = el;
		height = 1;
	}
	
	Node() {
		content = nullptr;
		height = 1;
	}
	
	Node& operator=(const Node& rhs) {
		L = rhs.L;
		R = rhs.R;
		content = rhs.content;
		height = rhs.height;
		return *this;
	}
};

int compare(char* str1, char* str2) {
	if (str1 == nullptr || str2 == nullptr) {
		if (str1 == nullptr && str2 == nullptr) return 0;
		return (str1 == nullptr) ? -1 : 1;
	}
	return strcmp(str1, str2);// -1 --- <; 0 -- ==;  1 --- >;
}

int get_height(Node* p) {
	return p ? p->height : 0;
}

int check_balance(Node* p) {// -2 - left; 0 - norm; 2 - right;
	return (get_height(p->L) - get_height(p->R)) == 0 ? 0 : (get_height(p->L) - get_height(p->R));
}

void fix_height(Node* p) {
	int a = get_height(p->L);
	int b = get_height(p->R);
	p->height = (a > b ? a : b) + 1;
}

Node* rot_right(Node* p) {
	Node* q = p->L;
	p->L = q->R;
	q->R = p;
	fix_height(p);
	fix_height(q);
	return q;
}

Node* rot_left(Node* root) {
	Node* p = root->R;
	root->R = p->L;
	p->L = root;
	fix_height(root);
	fix_height(p);
	return p;
}

Node* balance(Node* root) {
	if ((root->L) or (root->R)) fix_height(root);
	if (check_balance(root) == -2) {
		if (check_balance(root->R) > 0) root->R = rot_right(root->R);
		return rot_left(root);
	}
	if (check_balance(root) == 2) {
		if (check_balance(root->L) < 0)
		root->L = rot_left(root->L);
		return rot_right(root);
	}
	return root;
}

Node* ADD_ELEMENT(Node* root, char* content) {
	if (root == nullptr) return new Node(content);
	if (compare(content, root->content) == -1) root->L = ADD_ELEMENT(root->L, content);
	else if (compare(content, root->content) == 1) root->R = ADD_ELEMENT(root->R, content);
	else cout << "ERROR! This element have already exist.\n";
	return balance(root);
}

void printTree(Node* p) {
	cout << endl;
	if (p == nullptr) {
		cout << "Tree is empty..." << endl;
		return;
	}
	queue<Node*> q;
	q.push(p);
	int k = 128;
	int levelNodes = 1;
	Node* current = nullptr;
	while (!q.empty()) {
		k = k / 2;
		levelNodes = q.size();
		for (int i = 0; i < levelNodes; ++i) {
			current = q.front();
			q.pop();
			if (current != nullptr) {
				cout << setw(k) << current->content;
				
				q.push(current->L);
				q.push(current->R);
			}
			else {
				cout << setw(k) << "-";
			}
		}
		cout << endl;
	}
}

bool search_element(char* elem, Node* root) {
	cout << "Searching:\n";
	Node* cur = root;
	while (cur != nullptr) {
		if (compare(elem, cur->content) == -1) {
			cur = cur->L;
			cout << "left ";
		}
		else if (compare(elem, cur->content) == 1) {
			cur = cur->R;
			cout << "right ";
		}
		else if (compare(elem, cur->content) == 0){
			cout << "\nSearched: ";
			char loc = cur->content[0];
			int ind = 0;
			while (loc != '\0') {
				cout << loc;
				ind++;
				loc = cur->content[ind];
			}
			cout << "\n";
			return true;
		}
	}
	cout << "Element not found!!!!!\n";
	return false;
}

void MENU(Node* root) {
	cout << "Menu: " << endl;
	cout << "1-Exit" << endl;
	cout << "2-Insert element" << endl;
	cout << "3-Search element" << endl;
	cout << "4-Print tree" << endl;
	string el;
	int mode = 0;
	while (mode == 0) {
		cout << "Enter the number: \n";
		cin >> mode;
	}
	char* element = nullptr;
	if (mode == 1) exit(1);
	else if (mode == 2) {
		cout << "Enter the element:\n";
		cin >> el;
		char* element = new char[el.size()];
		for (int i = 0; i < el.size(); i++) {
			element[i] = el[i];
		}
		element[el.size()] = '\0';
		root = ADD_ELEMENT(root, element);
		printTree(root);
		MENU(root);
	}
	else if (mode == 3) {
		cout << "Enter the element for searching:\n";
		cin >> el;
		char* element = new char[el.size()];
		for (int i = 0; i < el.size(); i++) {
			element[i] = el[i];
		}
		element[el.size()] = '\0';
		search_element(element, root);
		MENU(root);
	}
	else if (mode == 4) {
		printTree(root);
		MENU(root);
	}
}

int main()
{
	setlocale(LC_ALL, "rus");
	Node* root = nullptr;
	MENU(root);
}
\end{lstlisting}


\newpage
\section{Тестирование программы}
В таблице \ref{tab1} представлены описания тестов по методологии чёрного ящика, все
тесты пройдены успешно.

\begin{center}
	
	\begin{table}[h!]
		\caption{Тестирование программы}
		\label{tab1}
		\centering
		\begin{tabular}{|m{3cm}|m{1.5cm}|m{5cm}|m{5cm}|}
			
			\hline
			Описание тестирования & 
			Входные данные & 
			Ожидаемый   результат &
			Полученный    результат \\ 
			\hline
			
			Проверка на корректность добавления элементов &
			
			a \newline
			s \newline
			d \newline
			g \newline
			h \newline
			j \newline
			abc &
			                  h\newline
				   d          s\newline 
			   a     g   j    -\newline
			-   abc - - - - - -\newline
			 &
			h\newline
			d          s\newline
			a     g   j    -\newline
			-   abc - - - - - -\newline \\
			\hline
			
			Проверка на поиск элемента в дереве &
			3 \newline
			a &
			left left
			Searched: a &
			left left
			Searched: a\\
			\hline
			
			Проверка на поиск несуществующего элемента элемента &
			3 \newline
			l &
			right left right Element not found!!!!! &
			right left right Element not found!!!!!\\
			\hline
			
		\end{tabular}
		
	\end{table}
	
\end{center}
\section*{Вывод}
Было реализовано \textbf{АВЛ-дерево}, а также было проведено тестирование работы программы. Все тесты были пройдены успешно.
\clearpage
