\chapter{Технологическая часть}

\section{Выбор средств реализации}
Для программной реализации алгоритма использовалась среда разработки Visual Studio 2022, язык программирования, на котором была выполнена реализации алгоритмов --- C++.
Для компиляции кода использовался компилятор MSVC. Исследование проводилось на ноутбуке (64--разрядная операционная система, процессор x64, частота процессора 3.1~ГГц, модель процессора 12th Gen Intel(R) Core(TM) i5-12500H, оперативная память 16~ГБ)
\section{Реализация алгоритмов}
В листингах \ref{list1}, \ref{list2}, \ref{list3}, \ref{list4}, \ref{list5} представлена программная реализация описанных классов и функций.
\begin{lstlisting}[label = list1, caption = Программная реализация стека и его интерфейса]
	template<typename tp>
	struct Stack {
		int capacity = 0;
		int col_elem = 0;
		tp* array = nullptr;
		int end = 0;
		int start = 0;
		
		Stack(int capacity) {
			this->capacity = capacity;
			array = new tp[capacity];
			for (int i = 0; i < capacity; i++) {
				this->array[i] = tp();
			}
		}
	};
	
	template<typename tp>
	bool IsEmpty(Stack<tp>* ds) {
		if (ds->col_elem == 0) return true;
		return false;
	}
	
	template<typename tp>
	bool Pop(Stack<tp>* stack, tp* contain) {
		if (IsEmpty(stack)) return false;
		*contain = stack->array[stack->end];
		stack->array[stack->end] = tp();
		stack->end -= 1;
		//cout << "Number " << (*contain) << " extracted from stack" << endl;
		stack->col_elem -= 1;
		return true;
	}
	
	template<typename DS>
	bool IsFull(DS* ds) {
		if (ds->capacity == ds->col_elem) return true;
		return false;
	}
	
	template<typename tp>
	bool Push(Stack<tp>* st, tp elem) {
		if (IsFull(st)) {
			cout << "Array is full" << endl;
			return 0;
		}
		for (int i = 0; i < st->capacity; i++) {
			if (st->array[i] == tp()) {
				if (IsEmpty(st)) {
					//cout << "Stack was empty" << endl;
					st->col_elem += 1;
					st->end = i;
					st->array[i] = elem;
					return 1;
				}
				//cout << "Stack was NOT empty" << endl;
				st->col_elem += 1;
				st->end = i;
				st->array[i] = elem;
				return 1;
			}
		}
		return 0;
	}
	
	float Peek(Stack<float>* st) {
		return st->array[st->end];
	}
	
	int Peek(Stack<int>* st) {
		return st->array[st->end];
	}
	
	string Peek(Stack<string>* st) {
		return st->array[st->end];
	}
	
	char Peek(Stack<char>* st) {
		return st->array[st->end];
	}
	
	template<typename tp>
	void print(Stack<tp>* st) {
		tp el;
		for (int i = 0; i < (st->end + 1); i++) {
			el = *(st->array[i]);
			cout << "[" << i + 1 << "]" << " --- " << el << endl;
		}
		cout << st->col_elem << " --- elemnts" << endl;
	}
	
	template<typename tp>
	bool Del(Stack<tp>* stack) {
		if (IsEmpty(stack)) return false;
		stack->array[stack->end] = tp();
		stack->end -= 1;
		stack->col_elem -= 1;
		return true;
	}
\end{lstlisting}

\begin{lstlisting}[label = list2, caption = Программная реализация функции валидации строки]
bool check_valid(string* str) {
	
	// lower
	int len = (*str).size();
	for (int i = 0; i < len; i++) {
		(*str)[i] = tolower((*str)[i]);
	}
	
	// checking the brackets
	int count = 0;
	for (int i = 0; i < len; i++) {
		if ((*str)[i] == '(') count += 1;
		if ((*str)[i] == ')') count -= 1;
		if (count < 0) return false;
	}
	
	
	string tg = "tg";
	string _3[4] = { "sin", "cos", "ctg", "atg" };
	string _4[3] = { "asin", "acos", "actg" };
	
	
	for (int i = 0; i < len; i++) {
		char elem = (*str)[i];
		if (is_num(elem)) {
			if (i == 0) {
				if (!((is_num((*str)[1]) or is_oper((*str)[1])))) {
					if (((*str)[1] == '(') or ((*str)[1] == '.')) continue;
					cout << "Unable symbol next to the first number\nTry again\n";
					return false;
				}
			}
			else if (i == (len - 1)) {
				if (!((is_num((*str)[len - 2]) or is_oper((*str)[len - 2])))) {
					if (((*str)[len - 2] == '(') or ((*str)[len - 2] == '.')) continue;///////////
					cout << "Unable symbol next to the last number\nTry again\n";
					return false;
				}
			}
			else {
				if (!((is_num((*str)[i - 1]) or is_oper((*str)[i - 1])) or ((*str)[i - 1] == '(')) or (!((is_num((*str)[i + 1]) or is_oper((*str)[i + 1]) or ((*str)[i + 1] == ')'))))) {
					if ((*str)[i - 1] == '.' or (*str)[i + 1] == '.') continue;
					cout << "Unable symbol next to the number\nTry again\n";
					return false;
				}
				continue;
			}
		}
		else if (is_oper(elem)) {
			
			if ((i + 1) >= len) {
				cout << "!!!'" << elem << "' can't be the last symbol!!!\nTry again\n";
				return false;
			}
			
			if ((i + 1) < len) {
				if (is_oper((*str)[i + 1])) {
					cout << "!!!check the the binary operator repetition!!!\n End try again\n";
					return false;
				}
			}
			
			if (elem == '-') {
				if ((i + 1) >= len) {
					cout << "!!!'-' can't be the last symbol!!!\nTry again\n";
					return false;
				}
				continue;
			}

			if ((i == 0)) {
				cout << "Binary operator can't be the first\nEnd try again\n";
				return false;
			}
			else {
				if (((*str)[i - 1] == '(') or ((*str)[i + 1] == ')')) {
					cout << "Check the correction of arguments for binary operator\nEnd try again\n";
					return false;
				}
				else if ((elem == '/') and ((*str)[i + 1] == '0')) {
					cout << "division by zero\nTry again\n";
					return false;
				}
			}
			
			
		}
		else if (is_brack(elem)) continue;
		else {
			bool indic = false;
			for (const auto [oper, od] : operators) 
			if (oper[0] == elem) {
				if ((i + oper.size()) > len) {
					continue;
				}
				string cur_func = "";
				for (int j = 0; j < oper.size(); j++) {
					cur_func += (*str)[i + j];
				}
				if (!(cur_func == oper)) {
					continue;
				}
				cur_func += (*str)[i + oper.size()];
				if (cur_func != (oper + '(')) {
					cout << "!!!You must write the '(' after functions!!!\nTry again\n";
					return false;
				}
				else {
					i += (oper.size() - 1);
					indic = true;
					break;
				}
			}
			if (indic) continue;
			cout << "!!!Such a function does not exist!!!\nTry again\n";
			return false;
		}
	}
	return true;
}
\end{lstlisting}

\begin{lstlisting}[label = list3, caption = Программная реализация функции токенезации]
string* create_arr(string* str, int* ln) {
	int len = (*str).size();
	
	int ind_arr = 0;
	string* arr = new string[len];
	
	string cur = "";
	for (int i = 0; i < len; i++) {
		cur += (*str)[i];
		if (is_oper((*str)[i])) {
			if (i == 0) arr[ind_arr] = "-u";
			else if ((i > 0) and ((*str)[i - 1] == '(')) arr[ind_arr] = "-u";
			else arr[ind_arr] = cur;
			cur = "";
			ind_arr++;
		}
		else if (is_num((*str)[i])) {
			if (len > (i + 1)) {
				if (!(is_num((*str)[i + 1]))) {
					arr[ind_arr] = cur;
					ind_arr++;
					cur = "";
				}
			}
			else {
				arr[ind_arr] = cur;
				ind_arr++;
				cur = "";
			}
		}
		else if (is_brack((*str)[i])) {
			arr[ind_arr] = cur;
			ind_arr++;
			cur = "";
		}
		else {
			if (len > (i + 1)) {
				if ((*str)[i + 1] == '(') {
					arr[ind_arr] = cur;
					ind_arr++;
					cur = "";
				}
			}
			else {
				arr[ind_arr] = cur;
				ind_arr++;
				cur = "";
			}
		}
	}
	(*ln) = ind_arr;
	return arr;
}
\end{lstlisting}

\begin{lstlisting}[label = list4, caption = Программная реализация функции превода в постфиксную запись]
string* create_postf_arr(string* st_arr, int *len) {
	Stack<string> stack(*len);
	string* postf_arr = new string[(*len)];
	
	int ind = 0;
	
	for (int i = 0; i < (*len); i++) {// i - cursor
		//1.
		if (is_num(st_arr[i])) {
			postf_arr[ind] = st_arr[i];
			ind++;
		}
		//2.
		else if ((st_arr[i]) == "(") {
			Push(&stack, st_arr[i]);
		}
		//3.
		else if ((st_arr[i]) == ")") {
			while (!(IsEmpty(&stack)) and (Peek(&stack) != "(")) {
				Pop(&stack, &((postf_arr)[ind]));
				ind++;
			}
			if (Peek(&stack) == "(") {
				Del(&stack);
			}
			//3.2
			else if (IsEmpty(&stack)) {
				break;
			}
		}
		//4.
		else {
			while (true) {
				//4.1
				if (IsEmpty(&stack)) {
					Push(&stack, st_arr[i]);
					break;
				}
				//4.2
				else if ((Peek(&stack) == "(")) {
					Push(&stack, st_arr[i]);
					break;
				}
				//4.3
				else if (ord(st_arr[i]) > ord(Peek(&stack))) {
					Push(&stack, st_arr[i]);
					break;
				}
				//4.4
				else {
					Pop(&stack, &((postf_arr)[ind]));
					ind++;
				}
			}
		}
	}
	
	while (!(IsEmpty(&stack))) {
		Pop(&stack, &((postf_arr)[ind]));
		ind++;
	}
	(*len) = ind;
	return postf_arr;
}
\end{lstlisting}

\begin{lstlisting}[label = list5, caption = Программная реализация функции вычисления выражений в постфиксной записи]
void calculate(string s, float left, float right, float step) {
	while ((left > right) || (step <= 0)){
		cout << "Enter correct dates" << endl;
		cout << "LEFT: ";
		cin >> left;
		cout << "RIGHT: ";
		cin >> right;
		cout << "STEP: ";
		cin >> step;
	}
	cin.ignore();
	cout << "Enter the arithmetic expression: ";
	getline(std::cin, s);
	string s_read;
	for (int i = 0; i < s.size(); i++) {
		if (s[i] != ' ') s_read += s[i];
	}
	int ans = check_valid(&s_read);
	while (!(ans)) {
		getline(std::cin, s);
		s_read.clear();
		for (int i = 0; i < s.size(); i++) {
			if (s[i] != ' ') s_read += s[i];
		}
		ans = check_valid(&s_read);
	}
	
	int len;
	string* arr = create_arr(&s_read, &len);
	
	string* post_arr = create_postf_arr(arr, &len);
	cout << endl;
	cout << "LEN: " << len << endl;
	cout << "POSTFIX: ";
	for (int i = 0; i < len; i++) {
		cout << post_arr[i] << "_";
	}
	cout << endl;
	
	Stack<float> nums(len);
	cout << "Value in the interval from " << left << " to " << right << " with step " << step << ":";
	for (float i = left; i <= right; i += step) {
		if (i >= right) i = right;
		int ind = 0;
		for (int j = 0; j < len; j++) {
			if (is_num(post_arr[j])) {
				if (post_arr[j] == "x") {
					Push(&nums, i);
				}
				else {
					Push(&nums, stof(post_arr[j]));
				}
			}
			else {
				if (nums.col_elem < operands[post_arr[j]]) {
					cout << "!!!Insufficient number of arguments!!!";
					exit(1);
				}
				else {
					if (operands[post_arr[j]] == 1) {
						float _1;
						Pop(&nums, &_1);
						Push(&nums, arithm(_1, post_arr[j]));
					}
					else {
						float _1, _2;
						Pop(&nums, &_1);
						Pop(&nums, &_2);
						Push(&nums, arithm(_1, _2, post_arr[j]));
					}
				}
			}
		}
		float ans;
		Pop(&nums, &ans);
		cout << endl;
		cout << "ANSWER(x = " << i << "): " << ans << endl;
	}
}
\end{lstlisting}

\clearpage
