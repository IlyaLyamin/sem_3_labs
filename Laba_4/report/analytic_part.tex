\chapter{Аналитическая часть}
\section{Схема Дженнингса}
Схема Дженнингса -- это схема для сжатого хранения квадратных симметричных матриц. Наиболее эффективно это схема работает с матрицами, элементы которых сгруппированы около главной диагонали. Сжатая матрица  представляется в виде двух массивов (AN, D). AN -- массив для хранения значимых элементов, а D -- массив для хранения координат элементов главной строки относительно массива AN. Значимыми элементами называем все элементы главной диагонали, все ненулевые элементы, расположенные в верхем треугольнике матрицы, и все такие нули, после которых, при просмотре вдоль строки, находятся ненулевые элементы.

Таким образом, вместо хранения двумерного массива из $n^{2}$ элементов, будет храниться два 
массива, первый из $m$ элементов, где $m$ -- количество значимых элементов, и второй массив длины n.

Проблема этой схемы это не эффективное хранение матриц с элементами на побочной диагонали.
\section{Кольцевая схема Рейнбольдта-Местеньи}
Схема Рейнбольдта-Местеньи -- это схема для сжатого хранения произвольных матриц. Сжатая матрица представляется в виде пяти массивов (AN, NR, NC, JR, JC). AN -- массив для хранения значимых элементов, JR -- массив для хранения координат, относительно массива AN, элементов которые являются первыми в соответствующей строке, JC -- массив для хранения координат, относительно массива AN, элементов которые являются первыми в соответствующем столбце. NC -- массив для хранения индексов, в массиве AN, следующего элемента в с столбце изначальной матрицы. NR -- массив для хранения индексов, в массиве AN, следующего элемента в с строке изначальной матрицы. Значимыми элементами называем все ненулевые элементы. Для такой схемы хранения, меньшее количество элементов изначальной матрицы являются значимыми, что гарантирует одинаковое количество занимаемой памяти, что диагональной матрицей, что матрицей с элементами на побочной диагонали, а это и есть проблема сжатия по Дженнингсу.
\clearpage
