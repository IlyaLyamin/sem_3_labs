\chapter{Технологическая часть}

\section{Выбор средств реализации}
Для программной реализации алгоритма использовалась среда разработки Visual Studio 2022, язык программирования, на котором была выполнена реализации алгоритмов --- C++.
Для компиляции кода использовался компилятор MSVC. Исследование проводилось на ноутбуке (64--разрядная операционная система, процессор x64, частота процессора 3.1~ГГц, модель процессора 12th Gen Intel(R) Core(TM) i5-12500H, оперативная память 16~ГБ)
\section{Реализация алгоритмов}
В листингах \ref{list1}, \ref{list2}, \ref{list3}, \ref{list4}, \ref{list5}, \ref{list6}, \ref{list7}, \ref{list8}, \ref{list9}представлена программная реализация описанных классов и функций.
\begin{lstlisting}[label = list1, caption = Программная реализация вспомогательных структур]
class Matrix {
	public:
	int rows;
	int columns;
	int un_zero = 0;
	int** array;
	
	Matrix() : rows(0), columns(0), array(nullptr) {}
	
	Matrix(int size) {
		this->rows = size;
		this->columns = size;
		array = new int* [size];
		for (int i = 0; i < size; i++) {
			array[i] = new int[size];
			for (int j = 0; j < size; j++) {
				array[i][j] = 0;
			}
		}
	}
	
	Matrix(int rows, int columns) {
		this->rows = rows;
		this->columns = columns;
		array = new int* [rows];
		for (int i = 0; i < rows; i++) {
			array[i] = new int[columns];
			for (int j = 0; j < columns; j++) {
				array[i][j] = 0;
			}
		}
	}
	
	~Matrix() {
		clear();
	}
	
	void print() {
		for (int i = 0; i < rows; i++) {
			for (int j = 0; j < columns; j++) {
				cout << setw(4) << array[i][j] << " ";
			}
			cout << "\n";
		}
		cout << "ROWS: " << rows << "\n";
		cout << "COLUMNS: " << columns << "\n";
		cout << "UN_ZERO: " << un_zero << "\n";
	}
	
	Matrix& operator= (const Matrix& rhs) {
		if (this == (&rhs)) {
			cout << "slefassignment" << "\n";
			return *this;
		}
		clear();
		
		this->un_zero = rhs.un_zero;
		this->rows = rhs.rows;
		this->columns = rhs.columns;
		array = new int* [rows];
		
		for (int i = 0; i < rows; i++) {
			array[i] = new int[columns];
			for (int j = 0; j < columns; j++) {
				array[i][j] = rhs.array[i][j];
			}
		}
		return *this;
	}
	
	void clear() {
		if (array != nullptr) {
			for (int i = 0; i < rows; i++) {
				delete[] array[i];
			}
			delete[] array;
			array = nullptr;
		}
	}
};

struct CompressedMatrix {
	int* AN = nullptr;
	int* D = nullptr;
	int len_AN = 0;
	int size = 0;
	
	CompressedMatrix() {}
	
	CompressedMatrix(int* AN, int* D, int len_AN, int size) {
		this->AN = new int[len_AN];
		for (int i = 0; i < len_AN; i++) {
			this->AN[i] = AN[i];
		}
		this->D = new int[size];
		for (int i = 0; i < size; i++) {
			this->D[i] = D[i];
		}
		this->len_AN = len_AN;
		this->size = size;
	}
	
	~CompressedMatrix() {
		clear();
	}
	
	void print() {
		cout << "AN: ";
		for (int i = 0; i < len_AN; i++) {
			cout << AN[i] << " ";
		}
		cout << endl << "D: ";
		for (int i = 0; i < size; i++) {
			cout << D[i] << " ";
		}
	}
	
	CompressedMatrix& operator= (const CompressedMatrix& rhs) {
		if (this == (&rhs)) {
			cout << "slefassignment" << "\n";
			return *this;
		}
		clear();
		
		this->size = rhs.size;
		this->len_AN = rhs.len_AN;
		this->AN = new int[len_AN];
		this->D = new int[size];
		
		for (int i = 0; i < len_AN; i++) {
			AN[i] = rhs.AN[i];
		}
		for (int i = 0; i < size; i++) {
			D[i] = rhs.D[i];
		}
		return *this;
	}
	
	void clear() {
		if (AN != nullptr) {
			delete[] AN;
			AN = nullptr;
		}
		if (D != nullptr) {
			delete[] D;
			D = nullptr;
		}
	}
};

class KRMCompressedMatrix {
	public:
	int* AN = nullptr;
	int* NR = nullptr;
	int* NC = nullptr;
	int* JR = nullptr;
	int* JC = nullptr;
	int len_AN = 0;
	int rows = 0;
	int columns = 0;
	
	KRMCompressedMatrix() {}
	
	KRMCompressedMatrix(Matrix* mat) {
		this->len_AN = mat->un_zero;
		this->rows = mat->rows;
		this->columns = mat->columns;
		AN = new int[len_AN];
		NR = new int[len_AN];
		NC = new int[len_AN];
		JR = new int[rows];
		JC = new int[columns];
		for (int i = 0; i < len_AN; i++) {
			AN[i] = 0;
			NR[i] = -1;
			NC[i] = -1;
		}
		for (int i = 0; i < rows; i++) {
			JR[i] = -1;
		}
		for (int i = 0; i < columns; i++) {
			JC[i] = -1;
		}
		create_arrays(mat, AN, NR, NC, JR, JC);
	}
	
	KRMCompressedMatrix(map<pair<int, int>, int>* coords, int rows, int columns) {
		this->len_AN = (*coords).size();
		this->rows = rows;
		this->columns = columns;
		
		AN = new int[len_AN];
		NR = new int[len_AN];
		NC = new int[len_AN];
		JR = new int[rows];
		JC = new int[columns];
		
		for (int i = 0; i < len_AN; i++) {
			AN[i] = 0;
		}
		for (int i = 0; i < rows; i++) {
			JR[i] = -1;
		}
		for (int i = 0; i < columns; i++) {
			JC[i] = -1;
		}
		
		create_arrays(coords, AN, NR, NC, JR, JC, rows, columns);
	}
	
	KRMCompressedMatrix(int len_AN, int rows, int columns, int*& AN, int*& NR, int*& NC, int*& JR, int*& JC) {
		this->len_AN = len_AN;
		this->rows = rows;
		this->columns = columns;
		this->AN = AN;
		this->NR = NR;
		this->NC = NC;
		this->JR = JR;
		this->JC = JC;
	}
	
	~KRMCompressedMatrix() {
		clear();
	}
	
	void print() {
		cout << "AN: ";
		for (int i = 0; i < len_AN; i++) {
			cout << setw(3) << AN[i] << " ";
		}
		
		cout << endl << "NR: ";
		for (int i = 0; i < len_AN; i++) {
			cout << setw(3) << NR[i] << " ";
		}
		
		cout << endl << "NC: ";
		for (int i = 0; i < len_AN; i++) {
			cout << setw(3) << NC[i] << " ";
		}
		
		cout << endl << "JR: ";
		for (int i = 0; i < rows; i++) {
			cout << setw(3) << JR[i] << " ";
		}
		
		cout << endl << "JC: ";
		for (int i = 0; i < columns; i++) {
			cout << setw(3) << JC[i] << " ";
		}
	}
	
	KRMCompressedMatrix& operator= (const KRMCompressedMatrix& rhs) {
		if (this == (&rhs)) {
			cout << "selfassignment" << "\n";
			return *this;
		}
		clear();
		
		this->rows = rhs.rows;
		this->columns = rhs.columns;
		this->len_AN = rhs.len_AN;
		this->AN = new int[len_AN];
		this->NR = new int[len_AN];
		this->NC = new int[len_AN];
		this->JR = new int[rows];
		this->JC = new int[columns];
		
		for (int i = 0; i < len_AN; i++) {
			this->AN[i] = rhs.AN[i];
			this->NR[i] = rhs.NR[i];
			this->NC[i] = rhs.NC[i];
		}
		for (int i = 0; i < rows; i++) {
			this->JR[i] = rhs.JR[i];
		}
		for (int i = 0; i < columns; i++) {
			this->JC[i] = rhs.JC[i];
		}
		return *this;
	}
	
	void clear() {
		if (AN != nullptr) {
			delete[] AN;
			AN = nullptr;
		}
		if (NR != nullptr) {
			delete[] NR;
			NR = nullptr;
		}
		if (NC != nullptr) {
			delete[] NC;
			NC = nullptr;
		}
		if (JR != nullptr) {
			delete[] JR;
			JR = nullptr;
		}
		if (JC != nullptr) {
			delete[] JC;
			JC = nullptr;
		}
	}
};
\end{lstlisting}

\begin{lstlisting}[label = list2, caption = Программная реализация вспомогательных функций]
Matrix usual_addition(Matrix* _1, Matrix* _2) {
	if ((_1->rows != _2->rows) or (_2->columns != _1->columns)) {
		cout << "Uncampatable sizes in addition: \n";
		exit(1);
	}
	Matrix ans(_1->rows, _1->columns);
	for (int i = 0; i < (*_1).rows; i++) {
		for (int j = 0; j < (*_2).rows; j++) {
			ans.array[i][j] = _1->array[i][j] + _2->array[i][j];
		}
	}
	return ans;
}

Matrix usual_multiplication(Matrix* array_l, Matrix* array_r) {
	Matrix new_ar(array_l->rows, array_r->columns);
	if (array_l->columns != array_r->rows) {
		cout << "Uncampatable sizes in multiplication: \n";
		exit(1);
	}
	
	for (int i = 0; i < array_l->rows; i++) {
		
		for (int j = 0; j < array_r->columns; j++) {
			for (int k = 0; k < array_l->columns; k++) {
				new_ar.array[i][j] += (array_l->array[i][k] * array_r->array[k][j]);
			}
			if (new_ar.array[i][j] != 0) new_ar.un_zero += 1;
		}
	}
	return new_ar;
}

int* parse(string str, int len, Matrix* mat) {
	int col = 0;
	for (char i : str)
	if (i == ';') col++;
	string* arr = new string[col];
	int iter = 0;
	for (char i : str) {
		if (i == ';') {
			iter++;
			continue;
		}
		arr[iter] += i;
	}
	if (iter == len) {
		int* ans = new int[iter];
		for (int i = 0; i < iter; i++) {
			ans[i] = str_int(arr[i]);
			if (ans[i] != 0) mat->un_zero += 1;
		}
		delete[] arr;
		return ans;
	}
	else {
		cout << "There are not enough numbers for a matrix of this size";
		exit(1);
	}
	
	Matrix read_matrix(string filename) {
		ifstream file(filename);
		string data;
		int rows = 0;
		int r = 0;
		int c = 0;
		if (file.is_open())
		{
			if (getline(file, data)) {
				if (data[0] == 'S') {
					read_len(data, &r, &c);
				}
			}
			Matrix matrix(r, c);
			// start reading the matrix
			while (getline(file, data)) {
				matrix.array[rows] = parse(data, c, &matrix);//////
				if (rows > r) {
					cout << "There are too many strings\n";
					exit(1);
				}
				rows++;
			}
			if (rows != r) {
				cout << "Amount of strings is not enoght\n";
				exit(1);
			}
			return matrix;
		}
		else {
			cout << "File read error\n";
			exit(1);
		}
	}
	
	void COMPARE(Matrix* _1, Matrix* _2) {
		for (int i = 0; i < _1->rows; i++) {
			for (int j = 0; j < _1->columns; j++) {
				if (_1->array[i][j] != _2->array[i][j]) {
					cout << "POSITOIN [" << i << "][" << j << "]: " << _1->array[i][j] << " != " << _2->array[i][j] << "\n";
				}
			}
		}
	}
\end{lstlisting}

\begin{lstlisting}[label = list3, caption = Программная реализация алгоритма упавковки по Дженнингсу]
	void create_AN(int** matrix, int size, int* an_size, int** D, int** AN) {
		vector<int> an_loc;
		
		int glob_ind = 0;
		int last_ind = 0;
		
		for (int i = 0; i < size; i++) {
			(*D)[i] = glob_ind;
			last_ind = i;
			for (int j = i; j < size; j++) {
				if (matrix[i][j] != 0) last_ind = j;
			}
			for (int j = i; j < (last_ind + 1); j++) {
				an_loc.push_back(matrix[i][j]);
			}
			glob_ind += (last_ind - i) + 1;
		}
		
		(*an_size) = an_loc.size();
		(*AN) = new int[(*an_size)];
		for (int i = 0; i < (*an_size); i++) {
			(*AN)[i] = an_loc[i];
		}
	}
	
	CompressedMatrix matrix_packagin(Matrix* mat) {
		int AN_size = 0;
		int* D = new int[(*mat).rows];
		int* AN = nullptr;
		
		create_AN((*mat).array, (*mat).rows, &AN_size, &D, &AN);
		
		CompressedMatrix A(AN, D, AN_size, (*mat).rows);
		delete[] AN;
		delete[] D;
		return A;
	}
\end{lstlisting}

\begin{lstlisting}[label = list4, caption = Программная реализация алгоритма распаковки из схемы Дженнингса]
Matrix matrix_unpackagin(CompressedMatrix* comp_mat) {
	Matrix ans(comp_mat->size);
	ans.un_zero = comp_mat->len_AN;
	cout << endl;
	
	int row = -1;
	int column = -1;
	int ind_D = 0;
	for (int i = 0; i < comp_mat->len_AN; i++) {
		column++;
		if (i == comp_mat->D[ind_D]) {
			row++;
			column = row;
			ind_D++;
		}
		ans.array[row][column] = comp_mat->AN[i];
		ans.array[column][row] = comp_mat->AN[i];
	}
	return ans;
}
\end{lstlisting}

\begin{lstlisting}[label = list5, caption = Программная реализация алгоритма сложения мартиц сажтаых по схеме Дженнингса]
CompressedMatrix sum_pack_matrix(CompressedMatrix* _1, CompressedMatrix* _2) {
	vector<int> *loc_sum = new vector<int>[_1->size];
	vector<int> result_AN;
	vector<int> result_D;
	int len_row_1 = 0;
	int len_row_2 = 0;
	int iter_row = 0;
	
	while (iter_row < (_1->size - 1)) {//size==rows
		len_row_1 = _1->D[iter_row + 1] - _1->D[iter_row];
		len_row_2 = _2->D[iter_row + 1] - _2->D[iter_row];
		for (int i = 0; i < max(len_row_1, len_row_2); i++) {
			loc_sum[iter_row].push_back(0);
			if (i < len_row_1) loc_sum[iter_row][loc_sum[iter_row].size() - 1] += _1->AN[_1->D[iter_row] + i];
			if (i < len_row_2) loc_sum[iter_row][loc_sum[iter_row].size() - 1] += _2->AN[_2->D[iter_row] + i];
		}
		iter_row++;
	}
	loc_sum[iter_row].push_back(0);
	loc_sum[iter_row][loc_sum[iter_row].size() - 1] += _1->AN[_1->D[iter_row]] + _2->AN[_2->D[iter_row]];
	int last_ind = 1;
	
	for (int i = 0; i < _1->size; i++) {
		result_D.push_back(result_AN.size());
		for (int el = 0; el < loc_sum[i].size(); el++) {
			if (loc_sum[i][el] != 0) last_ind = el;
		}
		for (int j = 0; j <= last_ind; j++) {
			result_AN.push_back(loc_sum[i][j]);
		}
	}
	
	int* D = new int[_1->size];
	int* AN = new int[result_AN.size()];
	for (int i = 0; i < result_AN.size(); i++) {
		AN[i] = result_AN[i];
	}
	for (int i = 0; i < _1->size; i++) {
		D[i] = result_D[i];
	}
	
	CompressedMatrix ans(AN, D, result_AN.size(), _1->size);
	return ans;
}
\end{lstlisting}

\begin{lstlisting}[label = list6, caption = Программная реализация алгоритма упаковки матрицы по схеме Рейнбольдта-Местеньи]
	int* get_IA(KRMCompressedMatrix* _1) {//rows numbers
		int* IA = new int[_1->len_AN];
		for (int i = 0; i < _1->len_AN; i++) {
			IA[i] = -1;
		}
		int cur;
		for (int i = 0; i < _1->len_AN; i++) {
			if (IA[i] == -1) {
				for (int r = 0; r < _1->rows; r++) {
					if (i == _1->JR[r]) {
						IA[i] = r;
						cur = _1->NR[i];
						while (cur != _1->JR[r]) {
							IA[cur] = r;
							cur = _1->NR[cur];
							//if (cur >= _1->len_AN) cout << "294!!!!!!!!!!!!!!!!!\n";
						}
					}
				}
			}
		}
		return IA;
	}
	
	int*& get_JA(KRMCompressedMatrix* _1) {// get columns numbers
		int* JA = new int[_1->len_AN];
		for (int i = 0; i < _1->len_AN; i++) {
			JA[i] = -1;
		}
		int cur;
		for (int i = 0; i < _1->len_AN; i++) {
			if (JA[i] == -1) {
				for (int c = 0; c < _1->columns; c++) {
					if (i == _1->JC[c]) {
						JA[i] = c;
						cur = _1->NC[i];
						while (cur != _1->JC[c]) {
							JA[cur] = c;
							cur = _1->NC[cur];
							//if (cur >= _1->len_AN) cout
						}
					}
				}
			}
		}
		return JA;
	}
	
	void create_arrays(Matrix* mat, int*& AN, int*& NR, int*& NC, int*& JR, int*& JC)
	{
		int R = mat->rows;
		int C = mat->columns;
		int LEN_AN = mat->un_zero;
		int iter = -1;
		int cur_R = 0;
		int cur_C = 0;
		int loc_iter = 0;
		for (int i = 0; i < R; i++) {
			for (int j = 0; j < C; j++) {
				if (mat->array[i][j] != 0) {
					iter++;
					if (iter >= LEN_AN) {
						exit(1);
					}
					AN[iter] = mat->array[i][j];
					NR[iter] = iter;
					NC[iter] = iter;
					
					if ((JR[i] == -1) or (JC[j] == -1)) {
						if (JR[i] == -1) JR[i] = iter;
						if (JC[j] == -1) JC[j] = iter;
					}
					//NR
					cur_R = JR[i];
					while (cur_R < iter) {
						NR[cur_R] = cur_R + 1;
						cur_R += 1;
					}
					NR[iter] = JR[i];
					
					//NC
					cur_C = JC[j];
					while (cur_C < iter) {
						if (NC[cur_C] == JC[j]) {
							NC[cur_C] = iter;
							cur_C = iter;
						}
						else if (NC[cur_C] != iter) {
							cur_C = NC[cur_C];
						}
						else {
							NC[cur_C] = iter;
							cur_C = NC[cur_C];//iter
						}
					}
					NC[iter] = JC[j];
				}
			}
		}
	}
	
\end{lstlisting}

\begin{lstlisting}[label = list7, caption = Программная реализация алгоритма распкаовки мартиц сажтаых по схеме Рейнбольдта-Местеньи]
	Matrix unpacking_KRM(KRMCompressedMatrix* krm) {
		int* IA = get_IA(krm);
		int* JA = get_JA(krm);
		
		Matrix res(krm->rows, krm->columns);
		for (int i = 0; i < krm->len_AN; i++) {
			res.array[IA[i]][JA[i]] = krm->AN[i];
		}
		res.un_zero = krm->len_AN;
		if (IA != nullptr) delete[] IA;
		if (JA != nullptr) delete[] JA;
		return res;
	}
\end{lstlisting}

\begin{lstlisting}[label = list8, caption = Программная реализация алгоритма сложения мартиц сажтаых по схеме Рейнбольдта-Местеньи]
	KRMCompressedMatrix KRM_addition(KRMCompressedMatrix* _1, KRMCompressedMatrix* _2) {
		if (_1->columns != _2->columns or _1->rows != _2->rows) {
			cout << "Uncampatable sizes in addition: \n";
			exit(1);
		}
		int rows = max(_1->rows, _2->rows);
		int columns = max(_1->columns, _2->columns);
		
		// coords arrays
		int* IA_1 = get_IA(_1);
		int* JA_1 = get_JA(_1);
		
		
		int* IA_2 = get_IA(_2);
		int* JA_2 = get_JA(_2);
		
		map <pair<int, int>, int> coords_3;
		for (int i = 0; i < _1->len_AN; i++) {
			if (coords_3.find({IA_1[i], JA_1[i]}) != coords_3.end()) {
				coords_3[{IA_1[i], JA_1[i]}] += _1->AN[i];
			}
			else {
				coords_3[{IA_1[i], JA_1[i]}] = _1->AN[i]; 
			}
		}
		
		for (int i = 0; i < _2->len_AN; i++) {
			if (coords_3.find({ IA_2[i], JA_2[i] }) != coords_3.end()) {
				coords_3[{IA_2[i], JA_2[i]}] += _2->AN[i];
			}
			else {
				coords_3[{IA_2[i], JA_2[i]}] = _2->AN[i];
			}
		}
		
		//removal the zeros
		for (auto it = coords_3.begin(); it != coords_3.end(); ) {
			if (it->second == 0) {
				it = coords_3.erase(it); // removal the element and renewing the iterator
			}
			else {
				++it; // move to the next
			}
		}
		
		int len_AN = coords_3.size();
		
		cout << "\n";
		
		KRMCompressedMatrix res(&coords_3, rows, columns);
		if (JA_1 != nullptr) delete[] JA_1;
		if (IA_1 != nullptr) delete[] IA_1;
		if (JA_2 != nullptr) delete[] JA_2;
		if (IA_2 != nullptr) delete[] IA_2;
		
		return res;
	}
\end{lstlisting}

\begin{lstlisting}[label=list9, caption = реализаия алгоритма умножения матриц сжатых по схеме Рейнбольдта-Местеньи]
	KRMCompressedMatrix KRM_multiplication(KRMCompressedMatrix* _1, KRMCompressedMatrix* _2) {
		int* IA_1 = get_IA(_1);
		int* JA_1 = get_JA(_1);
		
		// transponiruem
		int* IA_2 = get_IA(_2);
		int* JA_2 = get_JA(_2);
		
		map <pair<int, int>, int> coords_1;
		for (int i = 0; i < _1->len_AN; i++) {
			coords_1[{IA_1[i], JA_1[i]}] = _1->AN[i];
		}
		
		map <pair<int, int>, int> coords_2;
		for (int i = 0; i < _2->len_AN; i++) {
			coords_2[{IA_2[i], JA_2[i]}] = _2->AN[i];
		}
		
		int R = _1->rows;
		int C = _2->columns;
		
		int sum = 0;
		map <pair<int, int>, int> res_coords;
		for (int i = 0; i < R; i++) {
			for (int j = 0; j < C; j++) {
				for (int k = 0; k < _1->columns; k++) {
					if ((coords_1.find({i, k}) != coords_1.end()) and (coords_2.find({k, j}) != coords_2.end())) {
						sum += coords_1[{i, k}] * coords_2[{k, j}];
					}
				}
				// find sum
				if (sum != 0) {
					res_coords[{i, j}] = sum;
					sum = 0;
				}       
			}
		}
		
		KRMCompressedMatrix a(&res_coords, R, C);
		
		if (JA_1 != nullptr) delete[] JA_1;
		if (IA_1 != nullptr) delete[] IA_1;
		if (JA_2 != nullptr) delete[] JA_2;
		if (IA_2 != nullptr) delete[] IA_2;
		return a;
	}
\end{lstlisting}

\section{Тестирование программы}
В таблице~\ref{tab:tests1} и \ref{tab:tests2} представлены описания тестов по методологии чёрного ящика, все тесты пройдены успешно.
\begin{table}[htbp]
	\centering
	\caption{Описание тестов по методологии чёрного ящика}
	\begin{tabular}{|p{0.05\linewidth}|p{0.22\linewidth}|p{0.2\linewidth}|p{0.2\linewidth}|p{0.2\linewidth}|}
		\hline
		& \textbf{Описание теста} & \textbf{Входные данные} & \textbf{Ожидаемый результат} & \textbf{Полученный результат} \\
		\hline
		
		\textbf{1} 
		& проверка на обработку не валидных данных 
		& 3 
		& оповещение о некорректности данных и запрос новых данных 
		& оповещение о некорректности данных и запрос новых данных \\
		\hline
		
		\textbf{2} 
		& проверка на обработку не валидных данных 
		& 1 
		& включение режима работы с матрицами через схему Дженнингса  
		& включение режима работы с матрицами через схему Дженнингса \\
		\hline
		
		\textbf{3} 
		& проверка на обработку не валидных данных 
		& 2
		& включение режима работы с матрицами через схему Рейнбольдта-Местеньи 
		& включение режима работы с матрицами через схему Рейнбольдта-Местеньи \\
		\hline
		
		\textbf{4}
		& проверка на обработку не валидных данных в Дженнингсе 
		& SIZE;(2, 2);\newline 1;2;\newline 2;2;1; 
		& оповещение о некорректности данных и завершение программы 
		& оповещение о некорректности данных  и завершение программы \\
		\hline
		
		\textbf{5} 
		& проверка суммирования матриц в виде Дженнингса
		&Файл 1:\newline 1 0 0 0\newline 0 2 3 0\newline 0 3 4 0\newline0 0 0 5\newline Файл 2:\newline2 3 0 0\newline 3 0 0 4\newline0 0 1 0\newline 0 4 0 5
		& первая матрица упакованная:\newline AN: 1 2 3 4 5\newline D: 0 1 3 4\newline вторая матрица упакованная: \newline AN: 2 3 0 0 4 1 5\newline D: 0 2 5 6 \newline результирующая матрица упакованная: \newline AN: 3 3 2 3 4 5 10\newline D: 0 2 5 6 \newline результирующая матрица:\newline 3 3 0 0\newline
		3 2 3 4\newline 
		0 3 5 0\newline 
		0 4 0 10
		& первая матрица упакованная:\newline AN: 1 2 3 4 5\newline D: 0 1 3 4\newline вторая матрица упакованная: \newline AN: 2 3 0 0 4 1 5\newline D: 0 2 5 6 \newline результирующая матрица упакованная: \newline AN: 3 3 2 3 4 5 10\newline D: 0 2 5 6 \newline результирующая матрица:\newline 3 3 0 0\newline
		3 2 3 4\newline 
		0 3 5 0\newline 
		0 4 0 10 \\
		\hline
	\end{tabular}
	\label{tab:tests1}
\end{table}
\newpage
\begin{table}[htbp]
	\centering
	\caption{Описание тестов по методологии чёрного ящика}
	\begin{tabular}{|p{0.05\linewidth}|p{0.22\linewidth}|p{0.2\linewidth}|p{0.2\linewidth}|p{0.2\linewidth}|}
		\hline
		& \textbf{Описание теста} & \textbf{Входные данные} & \textbf{Ожидаемый результат} & \textbf{Полученный результат} \\
		\hline
		
		\textbf{6} 
		& проверка для схемы Дженнингса на сложение матриц с итоговой нулевой
		& Файл 1:\newline 0 10 0 \newline 0 0 0\newline 0 0 0\newline Файл 2:\newline 0 -10 0\newline 0 0 0\newline 0 0 0 
		& первая матрица упакованная:\newline AN: 0 10 0 0 \newline D: 0 2 3\newline вторая матрица упакованная: \newline AN:  0 -10 0 0\newline D: 0 2 3\newline результирующая матрица упакованная: \newline AN: 0 0 0\newline D: 0 1 2\newline результирующая матрица:\newline 0 0 0\newline
		0 0 0\newline 
		0 0 0
		&первая матрица упакованная:\newline AN: 0 10 0 0 \newline D: 0 2 3\newline вторая матрица упакованная: \newline AN:  0 -10 0 0\newline D: 0 2 3\newline результирующая матрица упакованная: \newline AN: 0 0 0\newline D: 0 1 2\newline результирующая матрица:\newline 0 0 0\newline
		0 0 0\newline 
		0 0 0
		\\
		\hline
		
		\textbf{7} 
		& проверка для КРМ на умножение матриц
		& Файл 1:\newline 0 10 0\newline 0 0 0\newline20 80 0\newline Файл 2:\newline 0 -10 0\newline0 0 0\newline -20 -80 0
		& 0 0 0\newline 0 0 0 \newline 0 -200 0 \newline упакованная результирующая матрица:\newline	AN: -200\newline NR: 0\newline NC: 0\newline JR: -1 -1 0\newline JC: -1 0 -1
		& 0 0 0\newline 0 0 0 \newline 0 -200 0 \newline упакованная результирующая матрица:\newline	AN: -200\newline NR: 0\newline NC: 0\newline JR: -1 -1 0\newline JC: -1 0 -1\\
		\hline
	\end{tabular}
	\label{tab:tests2}
\end{table}
\clearpage
