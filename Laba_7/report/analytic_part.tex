\chapter{Аналитическая часть}
\section{Система счислений в виде простых множителей}
Система счисления в виде множителей числа --- это способ представления чисел в виде произведения их простых множителей. Такой подход используется, например, в задачах, связанных с теорией чисел, криптографией или для специальных арифметических операций. Любое число представимо в виде произведения своих простых множителей, при этом такое представление единственно с точностью до перестановки множителей.

\begin{equation}
	N = p_1^{e_1}\cdot p_2^{e_2}\cdot...\cdot p_n^{e_n}
\end{equation}

Где $p_1, p_2,...,p_n$ -- простые множители числа, а  $e_1, e_2,...,e_n$ -- степени множителей.
Операции в такой системе:
\begin{itemize}
	\item умножение -- степени всех делителей суммируются,
	\item деление -- степени всех делителей вычитаются,
	\item сложение/вычитание -- при этих операция, сначала приводим общие множители(дел\-ители) и получаем число равное их произведению, множители обоих числе не попавших в общие перемножаем, множители первого перемножаем между собой и множители второго тоже между собой, и проводим соответствующую операцию(сложение/вычитание), полученное в результате этого действия число представляем в виде множителей и умножаем на число полученное от произведения всех общих множителей.
\end{itemize}

\section{Длинная арифметика}
\textbf{Длинная арифметика} -- это набор программных средств, позволяющий работать с числами гораздо больших величин, чем это позволяют стандартные типы данных. Операции в длинной арифметике выполняются поразрядно. 
\clearpage
